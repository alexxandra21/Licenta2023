\chapter{Implementarea problemei în Dafny}


\section{ Reprezentarea datelor de intrare și ieșire }
În cazul problemei discutate datele de intrare sunt reprezentate de un număr natural \textbf{sum} 
pentru care se verifică dacă se produce soluția finală optimă.

Drept date de ieșire avem soluția de cost minim reprezentată de secvența formată din 6 numere naturale: \par
$\bullet$ soluție  = { $b_{0}, b_{1}, b_{2}, b_{3}, b_{4}, b_{5}$} , unde $\sum_{k=0}^{5} b_{k} \cdot 2^{k} = suma $ ~\cite{jared:1}.

\section{ Predicate folosite care garantează optimalitatea soluției}
Soluția este reprezentată ca o secvență de numere naturale, fiind chiar secvența descrisă la secțiunea date de ieșire.
Fiecare element din soluție reprezintă numărul de apariții ale bancnotei corespunzătoare în soluție.\par
Entry point-ul verificării  problemei este metoda banknoteMinimum, care asigură faptul că soluția returnată este optimă.

$\bullet$ \textbf{predicatul isValidSolution}: Pentru a fi soluție optimă trebuie ca secvența să fie o soluție validă (cu 6 elemente), 
care produce suma corectă și care are costul cel mai mic.\par
\begin{lstlisting}
predicate isValidSolution(solution: seq < int > ) 
{
  hasSixElem(solution) && solution[0] >= 0 && solution[1] >= 0 && solution[2] >= 0 && solution[3] >= 0 && solution[4] >= 0 && solution[5] >= 0
}
\end{lstlisting}

$\bullet$ \textbf{predicatul isSolution}: Pentru a fi soluție trebuie ca secvența să fie o soluție validă (cu 6 elemente) și
să producă suma corectă.\par
\begin{lstlisting}
predicate isSolution(solution: seq < int > , sum: int)
  requires isValidSolution(solution) 
{
  solutionElementsSum(solution) == sum
}
\end{lstlisting}

$\bullet$ \textbf{predicatul isValidSolution}: Pentru a fi soluție validă trebuie ca secvența să aibă 6 elemente (tipuri de bancnote), cu 
proprietatea că fiecare bancnotă are un număr nul sau pozitiv de apariții în soluție.\par
\begin{lstlisting}
predicate isOptimalSolution(solution: seq < int > , sum: int)
  requires isValidSolution(solution) 
{
  isSolution(solution, sum) &&
    forall possibleSolution::isValidSolution(possibleSolution) && isSolution(possibleSolution, sum) ==> cost(possibleSolution) >= cost(solution)
}
\end{lstlisting}