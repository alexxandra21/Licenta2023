\chapter{Implementarea problemei în Dafny}


\section{ Reprezentarea datelor de intrare }
În cazul problemei discutate datele de intrare sunt reprezentate de o \textbf{sumă} 
pentru care se verifică dacă se produce soluția finală.

\section{ Reprezentarea datelor de ieșire}
Datele de ieșire sunt reprezentate de o secvență de forma : \par
$\bullet$ soluție  = { $b_{0}, b_{1}, b_{2}, b_{3}, b_{4}, b_{5}$} , unde $\sum_{k=0}^{5} b_{k} \cdot 2^{k} = suma $

\section{ Structuri de date folosite}
Soluția este reprezentată ca o secvență de numere naturale, fiind chiar secvența descrisă la secțiunea date de ieșire.
Fiecare element din soluție reprezintă numărul de apariții ale bancnotei corespunzătoare în soluție. 

\subsection{Condiții ca o soluție finală să fie optimă}
$\bullet$ \textbf{predicatul isValidSolution}: Pentru a fi soluție optimă trebuie ca secvența să fie o soluție validă (cu 6 elemente), 
care produce suma corectă și care are costul cel mai mic.\par
$\bullet$ \textbf{predicatul isSolution}: Pentru a fi soluție trebuie ca secvența să fie o soluție validă (cu 6 elemente) și
să producă suma corectă.\par
$\bullet$ \textbf{predicatul isValidSolution}: Pentru a fi soluție validă trebuie ca secvența să aibă 6 tipuri de bancnote, cu 
proprietatea că fiecare bancnotă are un număr nul sau pozitiv de apariții în soluție.\par
    