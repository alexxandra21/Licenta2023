\lstset{frame=tb,
	language=C++,
	aboveskip=3mm,
	belowskip=3mm,
	showstringspaces=false,
	columns=flexible,
	basicstyle={\small\ttfamily},
	numbers=none,
	numberstyle=\tiny,
	breaklines=true,
	breakatwhitespace=true,
	tabsize=3}

\chapter{Particularități ale problemei alese}
    
\section{Problema Bancnotelor cu bancnote puteri ale lui 2}
Anterior am menționat forma generală a Problemei Bancnotelor.\par
Bancnotele posibile în "Problema Bancnotelor cu bancnote puteri ale lui 2" sunt: 
$[1, 2, 4, 8, 16, 32]$ . 

\section{Observații}
    Datorită bancnotelor care sunt puteri ale lui 2, dacă avem mai mult de o bancnotă de valoare mai mică decât 32,
    putem înlocui 2 bancnote de acea valoare cu o bancnotă de valoarea următoare și am obține o soluție cu cost mai mic.\par
    Astfel, am descoperit proprietatea: 
    $  forall$ $i :: 0 <= i <= 4 ==> s[i] <= 1 $

