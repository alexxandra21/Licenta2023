\chapter{Implementarea problemei în Dafny}


\section{ Reprezentarea datelor de intrare }
În cazul problemei discutate datele de intrare sunt reprezentate de \textbf{suma} 
pentru care se verifică dacă se produce soluția finală optimă.

\section{ Reprezentarea datelor de ieșire}
Drept date de ieșire avem soluția de cost minim reprezentată de secvența: \par
$\bullet$ soluție  = { $b_{0}, b_{1}, b_{2}, b_{3}, b_{4}, b_{5}$} , unde $\sum_{k=0}^{5} b_{k} \cdot 2^{k} = suma $ \cite{jared:1}

\section{ Structuri de date folosite}
Soluția este reprezentată ca o secvență de numere naturale, fiind chiar secvența descrisă la secțiunea date de ieșire.
Fiecare element din soluție reprezintă numărul de apariții ale bancnotei corespunzătoare în soluție. 

\subsection{Condiții ca o soluție finală să fie optimă}
$\bullet$ \textbf{predicatul isValidSolution}: Pentru a fi soluție optimă trebuie ca secvența să fie o soluție validă (cu 6 elemente), 
care produce suma corectă și care are costul cel mai mic.\par
$\bullet$ \textbf{predicatul isSolution}: Pentru a fi soluție trebuie ca secvența să fie o soluție validă (cu 6 elemente) și
să producă suma corectă.\par
$\bullet$ \textbf{predicatul isValidSolution}: Pentru a fi soluție validă trebuie ca secvența să aibă 6 elemente (tipuri de bancnote), cu 
proprietatea că fiecare bancnotă are un număr nul sau pozitiv de apariții în soluție.\par
$\bullet$ \textbf{invariant}:Acest invariant, care apare mai jos, se asigură că 
nu avem în soluția optimă mai mult de o bancnotă de valoarea 1, 2, 4, 8 sau 16, deoarece aparițiile multiple ale acestora pot fi
înlocuite cu o bancnotă superioară pentru optimalitate(2 bancnote de valoare 1 au costul mai mare decât o bancnotă de valoare 2).\par
\begin{lstlisting}
    invariant forall index :: 4 >= index >= i ==> optimalSolution[index] <= 1
\end{lstlisting}

$\bullet$ \textbf{predicatul addOptimRestEqualsOptimSum}: Acest predicat are rolul de a verifica dacă o soluție pentru suma $x$ 
adunată cu o soluție optimă pentru suma $y$ are ca rezultat o soluție optimă pentru $x+y$. Acest lucru ne ajută să verificăm 
optimalitatea unei soluții fără să fie necesară optimalitatea ambelor soluții ce o formează.\par
Scopul acestor predicate este de a asigura că secvența ce reprezintă datele de ieșire respectă toate condițiile necesare
pentru a fi considerată soluție optimă.
    