\chapter{Reprezentarea problemei în Dafny}


\section{ Reprezentarea datelor de intrare }
În cazul problemei discutate datele de intrare sunt reprezentate de o variabilă \textbf{sum} ce reprezintă
 suma pentru care se verifică dacă se produce soluția finală.

\section{ Reprezentarea datelor de ieșire}
Datele de ieșire sunt reprezentate de o secvență de forma : \par
$\bullet$ soluție  = { $b_{0}, b_{1}, b_{2}, b_{3}, b_{4}, b_{5}$} , unde $\sum_{k=0}^{5} b_{k} \cdot 2^{k} = suma $

\section{ Structuri de date folosite}
-solutia reprezentata ca secventa
-cele 3 predicate explicate + exemple cod
\subsection{Condiții ca o soluție finală să fie optimă}
    $\bullet$ Pentru a avea o soluție optimă trebuie să avem o soluție validă (cu 6 elemente), care produce suma corectă
     și care are costul cel mai mic.\par
    $\bullet$ Pentru a avea o soluție optimă finală, pe parcursul construirii soluției, în buclă, trebuie menținută proprietatea
     de a alege soluția optimă locală pentru rest, iar suma soluțiilor optime locale să fie soluție optimă pentru sumă.\par
    $\bullet$ Soluția formată dintr-o bancnotă, cea aleasă în iterația curentă, produce o soluție optimă pentru suma 
    de valoare bancnotă, asigurând faptul că avem o soluție optimă locală.\par
    $\bullet$ O soluție optimă pentru suma $x$, adunată cu o soluție optimă pentru suma $y$ creează o soluție optimă pentru suma $x+y$, 
    asigurând faptul că suma soluțiilor locale creează soluția finală optimă.\par
    