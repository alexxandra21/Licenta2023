\chapter*{Concluzie} 
\addcontentsline{toc}{chapter}{Concluzii}

În cadrul acestei lucrări am realizat verificarea formală a unei implementări a metodei greedy ce rezolvă o variantă a 
problemei bancnotelor cu bancnote puteri ale lui 2.\par
Implementarea unei metode greedy ce rezolvă varianta clasică a problemei bancnotelor este deja existentă, în cadrul unei lucrări de 
licență prezentată în 2022 la Facultatea de Informatică Iași~\cite{elisa:1}.\par
Una dintre dificultățile întâlnite în acest proces a fost crearea invariantului care menține proprietatea de soluție
optimă, așa a apărut predicatul addOptimRestEqualsOptimSum.\par
Acest predicat afirmă că o soluție optimă pentru o sumă x adunată cu o soluție pentru suma y are ca rezultat o 
soluție optimă pentru suma $x+y$. Astfel, secvența ce reprezintă soluția care se construiește pe parcursul buclei 
are garanția că atât timp cât se adună cu soluții locale optime, în final, vom avea o soluție finală optimă.\par
Parcurgerea acestor etape pentru verificare m-a făcut să aprofundez noțiunea de soluție și să înțeleg în amănunt 
formarea acesteia prin metoda greedy.\par
Pe viitor îmi propun să găsesc o modalitate de a unifica metodele banknoteMaxim pentru a funcționa pentru orice caz, 
astfel optimizând viteza verificării.\par
Un alt lucru interesant de făcut în viitor, ar fi găsirea unei proprietăți care să se aplice pentru orice sistem 
de bancnote, pentru a verifica faptul că produce o soluție optimă.
 