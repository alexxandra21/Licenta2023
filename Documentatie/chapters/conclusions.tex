\chapter*{Pașii parcurși pentru finalizarea verificării} 
\addcontentsline{toc}{chapter}{Concluzii}

Pe parcursul verificării algoritmului am întâlnit o multitudine de erori și probleme la care știam exact ce anume nu 
funcționează cum m-am așteptat, ceea ce m-a adus la pasul de a rezolva problema, fără sa pierd timp căutând cauza ei.\par
Deoarece Dafny este un limbaj intuitiv și ușor de înțeles, pas cu pas m-am apropiat de rezultatul dorit, însă forma în care
această verificare se află azi nu este rezultatul la care mă așteptam inițial.\par
Inițial am abordat fiecare caz în parte într-un mod diferit, însă acest lucru a rezultat  într-un număr exagerat de 
if-uri. Pentru a testa că fiecare caz în parte nu este optim înafară de cel propus de algoritm, trebuia să parcurg
toate posibilitățile de a forma o soluție și să demonstrez că nu au cost minim. \par
Din dorința de a optimiza metoda de verificare am observat proprietatea discutată anterior, toate soluțiile optime
au maxim câte o bancnotă 1,2,4,8 și 16.\par
Astfel, am început să caut o modalitate de a verifica în mod dinamic cat mai multe cazuri simultan și am ajuns la o
formă apropiată de cea curentă a verificării.\par
Ulterior am realizat că foloseam multe funcții cu scopuri similare și am redus numărul acestora.
De asemenea, am șters toate assert-urile care nu erau necesare, deoarece foloseam multe pentru a mă convinge că 
ceea ce implementam se verifica din aproape în aproape.\par
În final am ajuns la un sfert din liniile de cod care se verificau inițial. \par
Parcurgerea acestor pași m-a făcut să aprofundez noțiunea de soluție și să înțeleg în amănunt formarea acesteia prin
metoda Greedy.\par
Pe viitor îmi propun să caut, dacă există, o proprietate pentru o soluție optimă care să se aplice pentru orice sistem de bancnote.
 