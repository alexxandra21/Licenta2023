
\chapter{Context} 

 
Dafny este un limbaj de programare și verificare, capabil să verifice corectitudinea
funcțională a unui program.\par
Verificarea este posibilă datorită caracteristicilor specifice limbajului precum precondiții, postcondiții, 
invariante, ș.a.m.d. De asemenea, verificatorul Dafny are grijă ca adnotările făcute să se îndeplinească,
astfel acesta ne scapă de povara de a scrie cod fără erori, în schimbul scrierii de adnotări fără erori.
 
Problema bancnotelor are ca scop reprezentarea unei sume într-un număr minim posibil de bancnote.\par
Metoda greedy face alegerea cea mai bună la fiecare pas, construind soluția finală.
Verificarea formală in Dafny a problemei bancnotelor demonstrează faptul că soluția construită este optimă
pentru orice sumă dată ca input. \par
Reprezentarea folosită pentru soluție este :
$ banknote_{1}:= 1 < banknote_{2} < ... < banknote_{n} $. 

Anterior am menționat forma generală a problemei bancnotelor.\par
Bancnotele posibile în problema bancnotelor cu $2^{5}$ sistem de bancnote sunt: 
$[1, 2, 4, 8, 16, 32]$ . 

Datorită bancnotelor care sunt puteri ale lui 2, dacă avem mai mult de o bancnotă de valoare mai mică decât 32,
putem înlocui 2 bancnote de acea valoare cu o bancnotă de valoarea următoare și am obține o soluție cu cost mai mic.\par
Astfel, am descoperit proprietatea: 
$  forall$ $i :: 0 <= i <= 4 ==> s[i] <= 1 $

