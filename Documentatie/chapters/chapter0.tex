
\chapter{Context} 

 
În cadrul acestei lucrări am ales să implementez și să verific formal algoritmul greedy pentru \textbf{problema bancnotelor}. La fiecare pas se selectează 
bancnota cea mai mare și se adaugă în soluție. Astfel reprezentăm suma primită printr-un număr minim de bancnote.\par

Problema bancnotelor la intrare primește un număr natural ce reprezintă o sumă și o rezervă infinită de bancnote de valori nominale predefinite, 
în funcție de cazul particular ales. Problema determină numărul minim posibil de bancnote în care putem reprezenta
restul de valoarea sumei. La iesire, problema returnează o secvență formată din numerele corespunzătoare aparițiilor 
fiecărei bancnote.\par
Reprezentarea generală folosită pentru soluție în problema bancnotelor este:
$ banknote_{1} < banknote_{2} < ... < banknote_{n}$, unde dacă $ banknote_{1} = x$, avem x bancnote de valoare 1.\par
Cazul particular ales pentru a fi implementat în această lucrare este problema bancnotelor în care sistemul de bancnote este format din puteri ale lui doi, 
până la puterea a 5-a $(2^0, 2^1, 2^2, 2^3, 2^4, 2^5)$.\par
Bancnotele posibile în problema bancnotelor cu sistemul de bancnote menționat sunt: $[1, 2, 4, 8, 16, 32]$. 

Un algoritm ce folosește \textbf{metoda greedy} face alegerea cea mai bună locală la fiecare pas, 
construind soluția finală, astfel sperând că aceasta este soluția optimă. Această metoda generează de fiecare dată 
o soluție mult mai bună decât soluția cea mai costisitoare, dar nu oferă soluția optimă pentru orice problemă.\par
Exemple de probleme pentru care metoda greedy oferă soluția optimă: \par
$\bullet$Problema de selecție a activității\par
$\bullet$Problema bancnotelor\par
$\bullet$Codificarea Huffman\par

Verificarea formală in Dafny a algoritmului greedy care rezolvă problema bancnotelor demonstrează faptul că soluția construită este optimă
pentru orice sumă dată ca input. \par
Dafny este un limbaj de programare și verificare, care poate verifica corectitudinea
funcțională a unui program. Verificarea este posibilă datorită caracteristicilor specifice limbajului precum precondiții, postcondiții, 
invariante, ș.a.m.d. De asemenea, verificatorul Dafny se asigură că adnotările făcute se îndeplinesc,
astfel acesta nu se așteaptă la scrierea unui cod fără erori, ci la scrierea unor adnotări fără erori.~\cite{leino2010dafny}\par
În următoarele rânduri voi explica pe scurt cum funcționează Dafny, folosind un exemplu simplu de cod preluat dintr-un repository public de pe Github, care 
rezolvă problema Fibonacci.~\cite{dafny:1}.

\begin{lstlisting}
function fib (n: nat) : nat
{
  if n == 0 then 0
  else if n == 1 then 1
  else fib (n-1) + fib (n-2)
}

method Fib (n: nat) returns (b: nat)
  ensures b == fib(n);
{
  if (n == 0) { return 0; }
  var a := 0;
  b := 1;
  
  var i := 1;
  while (i < n)
    invariant i <= n
    invariant a == fib(i-1);
    invariant b == fib(i);
  {
    a, b := b, a + b;
    i := i + 1;
  }
}
\end{lstlisting}

În exemplul dat putem observa cum metoda Fib are o post condiție $ensures$. Această postcondiție, dacă se verifică,
ne asigură că metoda Fib întradevăr returnează un număr b, care este egal cu rezultatul funcției Fib, care primește ca parametru n.\par
Dacă am fi folosit un tip de date intreg pentru n, am fi putut avea și o precondiție care să stabilească că n este mai mare decât zero.\par
Un alt lucru specific limbajului Dafny este prezența celor trei invarianți din bucla while(liniile 17, 18, 19).
Acești invarianți au rolul de a se asigura că sunt îndeplinite acele condiții pentru tot parcursul buclei, deoarece Dafny 
nu poate demonstra de la o iterație la alta ce s-a realizat în buclă.\par


