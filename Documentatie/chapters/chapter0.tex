\chapter{Introducere}
\section{Motivație} 

Motivul pentru care am ales această temă de licență este pentru a-mi aprofunda cunoștiințele legate de paradigma de programare Greedy. \par
La prima întâlnire cu algoritmii de tip Greedy, în liceu, mi s-au părut o modalitate foarte simplă de a găsi o soluție optimă. \par
La următorul contact cu acești algoritmi, în cadrul facultății, la materia "Proiectarea algoritmilor", 
 am aflat că partea interesantă și puțin mai complicată a acestor algoritmi este demonstrarea faptului că o soluție este soluție optimă. \par 
Mi-am dorit o temă realizabilă, care îmi poate provoca gândirea logică.\par
Prin  alegerea acestei teme, am vrut sa înțeleg mai bine crearea soluțiilor și
mi-am dat provocarea de a găsi o metodă mai scurtă, care poate fi generalizată, pentru găsirea soluției optime.

\section{Intenție} 

În cadrul lucrării am trei obiective .\par
În primul rând, îmi doresc să ofer o scurtă introducere în paradigma de programare Greedy.\par
În al doilea rând, îmi doresc să prezint limbajul Dafny făcând o analiză a plusurilor și minusurilor.\par
În al treilea rând, îmi doresc să arat implementarea proprie a algoritmului Greedy pentru Problema Bancnotelor in limbajul Dafny.\par

La finalul acestei lucrări ar trebui să ofer o imagine de ansamblu 
asupra felului în care această paradigmă funcționează, în ce gen de probleme poate fi aplicată și cum poate fi
implementată și demonstrată in limbajul Dafny. 

\section{Structură} 

Lucrarea va fi fragmentată în felul următor: 
\begin{enumerate}
	\item Paradigma Greedy - în acest capitol voi descrie paradigma Greedy și voi oferi exemple de probleme care se rezolvă cu aceasta
	\item Problema Bancnotelor - în acest capitol voi detalia problema implementată pentru licență
	\item De ce Dafny ?  - în acest capitol voi oferi o introducere în limbajul Dafny și voi menționa avantaje și dezavantaje 
	\item Implementarea Problemei - în acest capitol voi prezenta algoritmul implementat
\end{enumerate}