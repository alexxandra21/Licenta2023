


\chapter{Context} 

\section{Algoritmul Greedy pentru Problema Bancnotelor} 
Problema Bancnotelor este o problemă care reprezintă o sumă dată cu un număr minim posibil de bancnote.\par
Metoda Greedy face alegerea cea mai bună la fiecare pas. Soluția finală este formată din acele alegeri.
În această lucrare voi detalia verificarea formală in Dafny a problemei bancnotelor.\par
Verificarea constă în demonstrarea faptului că algoritmul produce o soluție optimă pentru toate 
sumele date ca input. \par
Soluția optimă este reprezentată cu un set de bancnote denumite în general

$ banknote_{1}:= 1 < banknote_{2} < ... < banknote_{n} $. 


\section{Problema Bancnotelor cu bancnote puteri ale lui 2}
Anterior am menționat forma generală a Problemei Bancnotelor.\par
În această lucrare am realizat verificarea formală pentru Problema Bancnotelor cu bancnote puteri ale lui 2.\par
Bancnotele posibile în problema implementată sunt: 
$[1, 2, 4, 8, 16, 32]$ . 

\section{Limbajul Dafny} 
Dafny este un limbaj de programare și verificare, capabil să verifice corectitudinea
 funcțională a unui program.\par
Verificarea este posibilă datorită caracteristicilor limbajului precum precondiții, postcondiții, 
invariante, ș.a.m.d. De asemenea, verificatorul Dafny are grijă ca adnotările făcute să se îndeplinească,
astfel acesta ne scapă de povara de a scrie cod fără erori, în schimbul scrierii de adnotări fără erori.