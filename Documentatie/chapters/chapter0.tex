


\chapter{Context} 

\section{Algoritmul Greedy pentru Problema Bancnotelor} 
Problema Bancnotelor are ca scop reprezentarea unei sume într-un număr minim posibil de bancnote.\par
Metoda Greedy face alegerea cea mai bună la fiecare pas, construind soluția finală.
Verificarea formală in Dafny a problemei bancnotelor demonstrează faptul că soluția construită este optimă
pentru orice sumă dată ca input. \par
Reprezentarea soluției :
$ banknote_{1}:= 1 < banknote_{2} < ... < banknote_{n} $. 


\section{Problema Bancnotelor cu bancnote puteri ale lui 2}
Anterior am menționat forma generală a Problemei Bancnotelor.\par
Bancnotele posibile în "Problema Bancnotelor cu bancnote puteri ale lui 2" sunt: 
$[1, 2, 4, 8, 16, 32]$ . 

\section{Limbajul Dafny} 
Dafny este un limbaj de programare și verificare, capabil să verifice corectitudinea
 funcțională a unui program.\par
Verificarea este posibilă datorită caracteristicilor specifice limbajului precum precondiții, postcondiții, 
invariante, ș.a.m.d. De asemenea, verificatorul Dafny are grijă ca adnotările făcute să se îndeplinească,
astfel acesta ne scapă de povara de a scrie cod fără erori, în schimbul scrierii de adnotări fără erori.