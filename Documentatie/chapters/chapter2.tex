\chapter{Reprezentarea problemei în Dafny}
\section{Tipuri de date folosite}
Tipurile de date folosite pentru a declara variabilele necesare:\par
 $\bullet int $ - numere întregi necesare pentru variabile ce reprezintă suma, bancnota, ș.a.m.d.\par
 $\bullet seq <int>$ - stocarea diferitelor soluții.\par
 $\bullet nat $ - pentru calculul puterilor.\par

\section{ Reprezentarea datelor de intrare }
În cazul problemei discutate datele de intrare sunt reprezentate de o variabilă \textbf{sum} ce reprezintă
 suma pentru care se verifică dacă se produce soluția finală.

\section{ Reprezentarea datelor de ieșire}
Datele de ieșire sunt reprezentate de o secvență de forma : \par
$\bullet$ soluție  = { $b_{0}, b_{1}, b_{2}, b_{3}, b_{4}, b_{5}$} , unde $\sum_{k=0}^{5} b_{k} \cdot 2^{k} = suma $